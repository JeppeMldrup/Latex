\documentclass[12pt]{article}

\usepackage[utf8]{inputenc}
\usepackage{amsmath}
\usepackage{fancyhdr}
\usepackage{graphicx}
\usepackage{vmargin}
\usepackage{chemfig}
\usepackage{tikz}
\usepackage{pgfplots}

%%%%%%%%%%%%%%%%%%%%%%%%%%%%%%%%%%%%%%%%%%%%%%%%%%%%%%%%%%%%%%%%%%%%%%%%%%

\setmarginsrb{3 cm}{2.5 cm}{3 cm}{2.5 cm}{1 cm}{1.5 cm}{1 cm}{1.5 cm}
\pagestyle{fancy}
\fancyhf{}
\rhead{07-05-2018}
\chead{Sidste kemi aflevering}
\lhead{Jeppe Møldrup}
\rfoot{side \thepage}

%%%%%%%%%%%%%%%%%%%%%%%%%%%%%%%%%%%%%%%%%%%%%%%%%%%%%%%%%%%%%%%%%%%%%%%%%%

\begin{document}

\Large{\textbf{Sidste kemi aflevering}}
\normalsize

\section*{Opgave 1.}
$$Fe_2O_{3(s)} \ + \ C_{(s)} \rightarrow Fe_{(s)} \ + \ CO_{(g)}$$
\begin{enumerate}
  \item[a.] $$\underbrace{Fe_2}_{OT=6} \underbrace{O_{3(s)}}_{OT=-6} \ + \
  3 \underbrace{C_{(s)}}_{OT=0} \rightarrow 2 \underbrace{Fe_{(s)}}_{OT=0} \ + \
  3 \underbrace{C}_{OT=2}\underbrace{O_{(g)}}_{OT=-2}$$

  \item[b.] Den molare masse af jern er $55.85 \ \frac{g}{mol}$. Så jeg finder antallet af mol der er i 2 tons jern
  $$\frac{2000000 \ g}{55.85 \ \frac{g}{mol}}=35810.2 \ mol$$
  Da forholdet mellem carbon og jern i reaktionen er 2:3 skal jeg bare gange antallet med forholdet, så jeg går den anden vej med carbon
  $$35810.2 \ mol \cdot \frac{2}{3}=53715.3 \ mol$$
  Den molare masse af carbon er $12.01 \ \frac{g}{mol}$
  $$\frac{53715.3 \ mol}{12.01 \ \frac{g}{mol}}=645120.9 \ g \approx 0.65 \ ton$$
  Så for at producere omkring 2 tons jern skal der bruges omkring 0.65 ton carbon

  \item[c.] Da der bruges omkring 645 kg carbon til at lave omkring 2 ton jern, ville 900 kg carbon kunne lave over 2 ton, nok tættere på 3
\end{enumerate}

\section*{Opgave 2.}
I kemijournal 17 var formålet at vi skulle finde mængden af jern i ståluld.\\
Vi tog cirka 1g ståluld og opløste det i en stærk syre. Derefter tilføjede vi $KMnO_4$ som reagerer med jern til at danne et hvidt bundfald dråbevis.\\
Vi blev ved med at proppe det i opløsningen indtil vi kunne se den pinke farve fra $KMnO_4$ da den stopper med at reagerer med jern fordi der ikke er
mere jern at reagere med. Så ud fra det kan vi udregne hvor meget jern der var i opløsningen fordi vi ved hvor meget $KMnO_4$ vi har proppet i, dermed kan
vi udregne hvor mange procent af stålulden der er jern. Vi kan forklare at der er carbon i stålulden med opgave 1 fordi i processen for at lave jern bliver
der også brugt carbon og hvis der ikke er 100 procent ækvivelente mængder når jernen bliver fremstillet kan der være noget carbon til overs som ikke er
reageret og derfor forbliver i det endelige produkt, og derfor ville det ikke være 100 procent jern.

\section*{Opgave 3.}
$$\underbrace{Fe^{3+}_{(aq)}}_{OT=3} \ + \
\underbrace{SCN^{-}_{(aq)}}_{OT=-1} \rightleftharpoons
\underbrace{FeSCN^{2+}_{(aq)}}_{OT=2}$$
\begin{enumerate}
  \item[a.] Jeg starter med at finde $Y$ og se om den er lig med $K$
  $$Y=\frac{1.0 \cdot 10^{-4} \ M}{1.1 \cdot 10^{-4} \ M \cdot 0.010 \ M}=91 \ m^{-1}$$
  Så den er tæt på $K$ men den er lidt over, dvs. at $Y<K$ og derfor vil reaktionen være forskudt til højre

  \item[b.] Jeg antager at $[FeSCN^{2+}]=1.0 \cdot 10^{-4} \ M$.
  $$8.9 \cdot 10^{2} \ M^{-1}=\frac{1.0 \cdot 10^{-4} \ M}{x \cdot 0.01 \ M}=1.1 \cdot 10^{-5} \ M$$
  Så den mindste koncentration ville være $1.1 \cdot 10^{5} \ M$
\end{enumerate}

\section*{Opgave 4.}
I kemijournal 20 havde vi reaktionen fra Opgave 3 og lavede nogle forskellige indgreb og noterede resultaterne\\
f.eks. havde vi et glas hvor vi proppede mere $Fe^{3+}$ i, og resultatet var at reaktionen var forskudt til højre.
det stemmer også overens med le chateliers princip idet vores indgreb blev mindsket.\\
I et andet glas øgede vi koncentrationen af $SCN^{-}$ og vi fik samme resultat.

\end{document}
