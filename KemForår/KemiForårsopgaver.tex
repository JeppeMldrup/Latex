\documentclass[12pt]{article}

\usepackage[utf8]{inputenc}
\usepackage{amsmath}
\usepackage{fancyhdr}
\usepackage{graphicx}
\usepackage{vmargin}
\usepackage{chemfig}
\usepackage{tikz}
\usepackage{pgfplots}

%%%%%%%%%%%%%%%%%%%%%%%%%%%%%%%%%%%%%%%%%%%%%%%%%%%%%%%%%%%%%%%%%%%%%%%%%%

\setmarginsrb{3 cm}{2.5 cm}{3 cm}{2.5 cm}{1 cm}{1.5 cm}{1 cm}{1.5 cm}
\pagestyle{fancy}
\fancyhf{}
\rhead{13-04-2018}
\chead{Kemi Forårsopgaver}
\lhead{Jeppe Møldrup}
\rfoot{side \thepage}

%%%%%%%%%%%%%%%%%%%%%%%%%%%%%%%%%%%%%%%%%%%%%%%%%%%%%%%%%%%%%%%%%%%%%%%%%%

\begin{document}

\Large{\textbf{Kemi Forårsopgaver}}
\normalsize

\section*{Opgave 56}
Jeg starter ud med at finde pH for min pufferopløsning inden jeg tilsætter min opløsning af $NaOH$ med pufferligningen
$$pH = pK_s + log(\frac{n_b}{n_s})$$
Så jeg indsætter mine værdier
$$pH = 9.25 + log(1) = 9.25$$

Så udregner jeg hvor mange mol jeg har af $NaOH$, $NH^+_{4(aq)}$ og $NH_{3(aq)}$
$$n(NaOH)=0.02 \ L \cdot 0.1 \ M = 0.002 \ mol$$
$$n(NH^+_{4(aq)})=0.01 \ L \cdot 0.05 \ M = 0.005 \ mol$$
$$n(NH_{3(aq)})=0.01 \ L \cdot 0.05 \ M = 0.005 \ mol$$

Nu laver jeg et reaktionsskema hvor jeg udregner hvor meget af syren som kommer til at reagere med min opløsning af $NaOH$
Jeg antager at $NaOH$ reagerer fulstændig med vand

\begin{center}
  \begin{tabular}{c c c c c c c c}

    & $OH^{-}_{(aq)}$ & $+$ & $NH_{4(aq)}^+$ & $\rightarrow$ & $H_2O_{(l)}$ & $+$ & $NH_{3(aq)}$\\
    $n_{start}$ & $0.002 \ mol$ && $0.005 \ mol$ &&&& $0.005 \ mol$\\
    $n_{brugt}$ & $0.002 \ mol$ && $0.002 \ mol$ &&&& $0 \ mol$\\
    $n_{dannet}$ & $0 \ mol$ && $0 \ mol$ && $0.002 \ mol$ && $0.002 \ mol$\\
    $n_{slut}$ & $0 \ mol$ && $0.003 \ mol$ &&&& $0.007 \ mol$

  \end{tabular}
\end{center}

Nu kan jeg igen bruge pufferligningen bare med mine nye værdier for at finde pH efter jeg har tilsat min opløsning af $NaOH$
$$pH = 9.25 + log(\frac{0.007 \ mol}{0.003 \ mol}) = 9.62$$
Så efter jeg tilsætter opløsningen med $NaOH$ kan vi forvente at pH stiger til cirka $9.62$.
Det giver også mening at den vil stige da det er en base som vi tilsætter.

\section*{Opgave 60}
\begin{enumerate}
  \item[a.] Beregn opløsningens pH
Jeg starter med at beregne antallet at mol for de to stoffer vet at bruge ligningen
$$n = \frac{m}{M}$$
Hvor n er antal mol, m er massen og M er molare masse\\
Den molare masse for de to stoffer er henholdvis $60.05 \ \frac{g}{mol}$ og $136.08 \ \frac{g}{mol}$
$$n(CH_3COOH) = \frac{1.75 \ g}{60.05 \ \frac{g}{mol}} = 0.029 \ mol$$
$$n(CH_3COONa \cdot 3H_2O) = \frac{5.82 \ g}{136.08 \ \frac{g}{mol}} = 0.043 \ mol$$
Nu kan jeg bruge pufferligningen til at finde pH for min opløsning
$$pH = pK_s + log(\frac{c(B)}{c(S)}) = 4.76 + log(\frac{0.043 \ mol}{0.029 \ mol}) = 4.93$$
Så pH for pufferopløsningen er cirka $4.93$

  \item[b.] Beregn pH efter der bliver tilsat 5 mL 2.0 M $HCl$

Jeg stater med at finde hvor mol jeg har af $HCl$
$$n(HCl) = 0.005 \ L \cdot 2.0 \ M = 0.01 \ mol$$

Så opstiller jeg et reaktionsskema for reaktionen mellem basen og min tilsatte syre
Jeg antager at $HCl$ reagerer fulstændig med vand.

\begin{center}
  \begin{tabular}{c c c c c c c c}

    & $H_3O^{-}_(aq)$ & $+$ & $CH_3COO^{-}_{(aq)}$ & $\rightarrow$ & $H_2O_{(l)}$ & $+$ & $CH_3COOH_{(aq)}$\\
    $n_{start}$ & $0.01 \ mol$ && $0.043 \ mol$ &&&& $0.029 \ mol$\\
    $n_{brugt}$ & $0.01 \ mol$ && $0.033 \ mol$ &&&& $0 \ mol$\\
    $n_{dannet}$ & $0 \ mol$ && $0 \ mol$ && $0.01 \ mol$ && $0.01 \ mol$\\
    $n_{slut}$ & $0 \ mol$ && $0.033 \ mol$ &&&& $0.039 \ mol$

  \end{tabular}
\end{center}

Nu kan jeg bruge pufferligningen til at beregne pH for opløsningen efter jeg har tilsat opløsningen af $HCL$
$$pH = 4.76 + log(\frac{0.033 \ mol}{0.039 \ mol}) = 4.69$$
Så efter man tilsætter $HCl$ opløsningen falder pH til cirka $4.69$. Det giver også mening at pH ville falde da det er en syre vi tilsætter

\end{enumerate}

\end{document}
