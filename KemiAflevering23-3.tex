\documentclass[12pt]{article}

\usepackage[utf8]{inputenc}
\usepackage{amsmath}
\usepackage{fancyhdr}
\usepackage{graphicx}
\usepackage{vmargin}
\usepackage{chemfig}

%%%%%%%%%%%%%%%%%%%%%%%%%%%%%%%%%%%%%%%%%%%%%%%%%%%%%%%%%%%%%%%%%%%%%%%%%%

\setmarginsrb{3 cm}{2.5 cm}{3 cm}{2.5 cm}{1 cm}{1.5 cm}{1 cm}{1.5 cm}
\pagestyle{fancy}
\fancyhf{}
\rhead{Kemi Afleveringsopgaver}
\chead{23-03-2018}
\lhead{Jeppe Møldrup}
\rfoot{side \thepage}

%%%%%%%%%%%%%%%%%%%%%%%%%%%%%%%%%%%%%%%%%%%%%%%%%%%%%%%%%%%%%%%%%%%%%%%%%%

\begin{document}
\LARGE{\textbf{Kemi Afleveringsopgaver}}
\normalsize{}

\section{Opg. 1}
3-Chlorpropansyre har $pK_s$ på 4,0.
\begin{itemize}
  \item Tegn strukturformlen for 3-Chlorpropansyre
  \begin{center}
  \chemfig{Cl-[:30]-[:-30]-[:30](=[:90]O)-[:-30]OH}
  \end{center}

  \item Opskriv reaktionsskemaet for 3-Chlorpropansyres reaktion med vand
  \begin{center}
  \schemestart
  \small
  \chemfig{ClCH_2CH_2COOH_{(aq)}} \+ \chemfig{H_2O_{(l)}} \arrow{<=>}
  \chemfig{ClCH_2CH_2COO^{-}_{(aq)}} \+ \chemfig{H_3O^+_{(aq)}}
  \schemestop
  \end{center}
  \normalsize

  \item
\end{itemize}

\end{document}
