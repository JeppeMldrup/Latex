\documentclass{article}

\usepackage[utf8]{inputenc}
\usepackage{fancyhdr}
\usepackage{hyperref}

\pagestyle{fancy}
\rhead{AT 6}
\chead{Jeppe og Mads}
\lhead{10/10-2018}

\title{AT 6 - Atomreaktorer}
\author{Jeppe Møldrup og Mads Damgaard}

\renewcommand{\contentsname}{Indholdsfortegnelse}

\begin{document}

\maketitle
\pagebreak
\tableofcontents
\pagebreak

\section{Indledning}

Størstedelen af mennesker i verden tænker at energikilden i fremtiden skal være grøn og bæredygtig, da fossile brændstoffer som en ressource løber tør, og ødelægger miljøet. Under grønne og bæredygtige energityper tænker de solenergi, vindenergi, bølgeenergi og så videre. En af de grønne energityper som har måske et af de største energipotentialer er atomenergi. Selvom potentialet er der, hører man generelt ikke meget om det, og det virker slet ikke til at være oppe til diskussion som en valgmulighed i Danmark. I denne opgave vil vi se på den udvikling atomenergien har gennemgået, som et alternativ til den primitive type, der har sået frygt i mange mennesker efter Tjernobyl katastrofen.

\section{Problemformulering}

Hvordan blev atomkraft udviklet som fredelig energiteknologi  - og hvad er denne teknologis fremtid?

\section{Underspørgsmål 1}

Gør kort rede for atomkraftværkets udviklingshistorie

\subsection{Metode}

Til at besvare underspørgsmålet har vi kigget på historiske kilder, der dokumenterer  og beskriver de forskellige kraftværks former, som er blevet brugt, op igennem tiden. Ved udvælgelsen af de kilder vi har brugt til at dokumentere udviklingen af atomkraften har vi brugt kildekritik, for at sikre os at vores kilder er troværdige.

\subsection{Delkonklusion}

Atomkraft teknologiens udvikling kan splittes op i 4 generationer i forhold til de traditionelle uran kraftværker

Generation 1(1950’erne til 1960’erne)

Herunder det første kommercielle atomkraftværk, Calder Hall, som der blev bygget i Seascape, England i 1956. Calder Hall var af typen AGR(Advanced Gas-Cooled Reactor) hvor man bruger carbondioxid under tryk til at køle reaktoren ned og generere strømmen.

Generation 2(1970’erne til nu)

Denne generation tilhører hovedparten af de nuværende a-værker som er i brug i dag. De er bygget til at have en levetid på omkring 40 år.
Herunder er reaktortypen RBMK(High-power channel reactor) som er systemet der blev brugt i katastrofen i Tjernobyl i 1986. RBMK-reaktorer er den eneste type reaktor i verden der bruger grafit i kontrolstavene og vand som kølemiddel. Og som det viste sig i 1986 var der nogle design valg i RBMK-reaktoren som gjorde den meget farlig. En af disse farlige design valg kommer med at bruge vand som kølevæske, idet vand både er et bedre kølemiddel og et bedre neutron absorberende middel end damp. Det betyder at reaktoren har chance for at “løbe løbsk”. Hvis reaktoren bliver lidt varmere end arbejderne ville have kommer der flere dampbobler i vandet, hvilket vil sige at det nu køler værre og absorberer mindre neutroner, også kaldet positivt feedback.

Generation 3(1990’erne)

Forbedret design med bedre udnyttelse, bedre sikkerhed og længere levetid. Dog har man nu fundet ud af at den øgede udnyttelse medfører mere radioaktivt affald, og dermed er det sværere at opbevare.
Der findes kun nogle få af disse i Japan. Europas første generation 3 værker er under konstruktion i Finland og Frankrig.

Generation 4

Generation 4 er til brug i 2030’erne, og har en række nye teknologier men stadig de samme grundlæggende problemer som de tidligere generationer af uran atomkraftværker, dog bedre end de andre generationer økonomisk og i forhold til radioaktivt affald. De er i øjeblikket ikke en realitet, men blot teori.


\section{Underspørgsmål 2}

Hvilke alternativer inden for fredelig udnyttelse af atomenergi bliver der aktuelt forsket i med henblik på at forhindre  katastrofer (som f. eks. Tjernobyl)?

\subsection{Metode}

Vi har kigget på naturfagligt empiri, der fortæller om de alternative atomkraftværker, som der bliver forsket i på tidspunktet. Forskere har her udøvet en række eksperimenter, for at finde hvilke stoffer der udsender mest energi ved deres henfald

\subsection{Delkonklusion}

I 1986 skullle der køres et eksperiment i Tjernobyl kraftværket i Ukraine. Her skulle de bringe effekten ned til omkring 1000 megawatt. De startede med at sænke effekten til omkring 1600 megawatt, hvor den blev nødt til at blive i et par timer fordi der var brug for strømen, derefter fortsatte sænkningen. Problemet med det var at ved den lave effekt bliver der dannet nogle gasser som er meget neutron absorberende, og derfor sænker effekten af reaktoren drastigt, så arbejderne havde meget svært at holde reaktoren i gang. De endte med at få liv i den og bare køre eksperimentet ved omkring 200 megawatt. Herefter for at kunne udføre deres eksperimenter blev de nødt til at slå flere forskellige sikkerhedssystemer fra og trække kontrolstavene næsten helt ud. Resultatet blev at da de neutron absorberende gasser, som havde hæmmet reaktoren forsvandt, steg temperaturen i reaktoren meget drastisk meget hurtigt. På grunde af denne temperatur stigning løb reaktoren løbsk, som forklaret i underspørgsmål 1, og blev så varm at der skete flere dampeksplosioner i reaktoren, som endte med at sende en radioaktiv sky op i luften. Efter eksplosionen brændte reaktoren i en uge, hvor der også blev spredt mange radioaktive gasser.
I 1985 den 29. marts, blev der vedtaget en beslutning om ikke at inkludere atomenergi i Danmarks fremtidige energiplanlægning, også kaldet beslutningforslag nr. B 103. Beslutningen blev foreslået af S, SF, RV og VS. Denne beslutning sammen med Tjernobyl katastrofen, der skete cirka et år senere, har holdt debatten om atomkraftværker i Danmark stille.
Men som Rasmus Brygger kommer ind på i sin artikel “Kernekraft fortjener en ny chance” så sker der meget på de 30 år der er gået siden beslutningen om atomenergi i 1985. Lige nu bliver der f.eks. forsket i at bruge grundstoffet thorium i stedet for uran i selve kernen, men selve beslutningen er også baseret på datidens generation 2 reaktorer. Hvor i dag har vi generation 3 og planer for generation 4 reaktorer der er meget mere effektive, sikre og miljøvenlige. Og som Rasmus siger så er vind- og solenergi meget dyrt, mens kulkraftværker ikke er miljøvenlige eller vedvarende. Så Rasmus mener at a-værker tager det bedste fra begge sider, nemlig at det er vedvarende(Det kan holde de næste mange tusind år) og det er miljøvenligt, og at a-værker har fået et unfair dårligt ry fordi der var nogle problemer med tidlige a-værker.


\section{Underspørgsmål 3}

Hvad er fordelene og ulemperne ved en thorium reaktor i forhold til en traditionel uran reaktor?

\subsection{Metode}

For at undersøge hvad fordelene og ulemperne ved en thorium reaktor i forhold til den traditionelle uran reaktor, har vi fundet naturfaglig empiri omkring begge typer af reaktor. Derefter har vi sammenlignet information omkring begge typer, for at kunne deduktivt konkludere på fordelene og ulemperne ved thorium reaktoren i forhold til den klassiske uran reaktor.

\subsection{Delkonklusion}

Til fordel for thorium er der estimeret til at være omkring 3-4 gange så meget thorium i forhold til uran. En anden fordel ved thorium er at det er grønnere i den forstand at uran skaber rigtig mange “transuraniske” elementer (kan f.eks. være plutonium), som er den farligste del af det radioaktive affald fra et atomkraftværk. Det tager kun en neutronindfangning til at lave Uran-238 om til et transuranisk element, mens med thorium-232 tager det typisk hele 5 indfangninger før det er muligt. Derfor bliver der produceret væsentligt mindre transuraniske elementer i et thoriumkraftværk, og dermed mindre affald. Derudover kan thorium udnyttes meget mere end uran, i det at næsten alt thorium kan konverteres om til energi, mens det kun er omkring en tohundrededel af uran der kan konverteres til energi. Ydermere kan nogle thorium systemer bruge noget af den radioaktive affald fra U-værker til energiproduktion og i stedet lave affald med meget mindre halveringstider. Thorium kan heller ikke rigtig bruges til at lave atombomber ligesom uran kan, dette kan ses som et stor plus i dag. Under den kolde krig valgte man at bruge uran i stedet for thorium lige præcis fordi de allerede forskede i uran til brug i våben, derfor er der ikke blevet forsket i brug af thorium inden for våben videnskaben, og manglen på info på dette område fremstiller thorium kraftværker som værende mere fredelige.
Der er dog også nogle ulemper ved thorium reaktorer. F.eks bliver der skabt Uran-232 i en thorium reaktor. Uran-232 er en isotop med en meget kort halveringstid, og dens henfaldskæde producere nogle meget energiske gammastråler, som er meget penetrerende, derfor skal man have mere sikkerhed i håndtering og reproduktion af Uran-232 og dens henfald.

\section{Konklusion}

For at konkludere har vi fundet at atomkraftens har undergået en stor udvikling, fra det stadie som mange tænker på når de hører ordet atomkraft. De fleste tænker på Tjernobyl når de tænker på atomkraft og dermed også stor fare og usikkerhed, men  som fredelig energiteknologi har atomkraftværkerne udviklet sig meget og der er kommet flere nye generationer der har gjort chancen for endnu en ulykke til kun en teoretisk en. På den måde har atomkraften fået et unfair ry, hvor spørgsmålet om atomkraftværk som en mulig energikilde ikke længere bliver bragt op i dansk politik. Selvom det i dag er noget helt andet end hvad det var da man i 1985 indførte loven imod inkluderingen af atomenergi. Det er blevet meget sikrere, mere effektivt og mere miljøvenligt, og fremtiden ser kun endnu bedre ud for atomkraft. Med forskning indenfor thorium-reaktorer, med bedre sikkerhed, mindre radioaktivt affald og ingen sekundær udnyttelse inden for krigsføring. Derfor er det muligt at forestille sig at debatten om hvorvidt Danmark skal have atomkraftværker vil komme op igen engang i fremtiden

\section{Perspektivering}

Vi kan perspektivere dette AT6 forløb til AT3-, AT4 og AT5-forløbet metodemæssigt.
I disse AT-forløb benyttede vi metoder inden for faget fysik,
hvor vi har brugt naturfagligt empiri.
Dog brugte vi ikke fysik koblet med historie, og derfor kan vi
ikke drage en komplet perspektivering mellem forløbene.
Vi kan også perspektivere til AT2, hvor vi brugte Idræt C og Historie A. I AT2- 
og dette AT-forløb har vi anvendt historisk kildekritik på
en række kilder for at kunne bestemme deres validitet.

\section{Litteraturliste}
\small

Advantage and Disadvantages of Thorium Reactors – Pros and Cons. Udgivet af Nuclear Power. Internetadresse:\\
nuclear-power.net/nuclear-power-plant/reactor-types/thorium-reactor/advantage-and-disadvantages-of-thorium-reactors-pros-and-cons/\\
Besøgt d. 08.10.2018 (Internet)\\

Atomkraftens udvikling. Udgivet af Vedvarende Energi. Internetadresse:\\
ve.dk/atomkraftens-udvikling/\\
Besøgt d. 08.10.2018 (Internet)\\

B103 af 1985. Udgivet af Folketinget. Internetadresse:\\
ens.dk/sites/ens.dk/files/EnergiKlimapolitik/akraftbeslutning\_ 1985.pdf\\
Besøgt d. 08.10.2018 (Internet)\\

Calder Hall. Udgivet af European nuclear society . Internetadresse:\\
euronuclear.org/info/encyclopedia/calderhall.htm\\
Besøgt d. 08.10.2018 (Internet)\\

Hvad gik galt i Tjernobyl?. Udgivet af Ingeniøren. Internetadresse:\\
ing.dk/artikel/hvad-gik-galt-i-tjernobyl-14349\\
Besøgt d. 08.10.2018 (Internet)\\

Kernekraften bliver bedre, sikrere og renere. Udgivet af Illustreret Videnskab. Internetadresse:\\
illvid.dk/teknologi/energi/atomkraft/kernekraften-bliver-bedre-sikrere-og-renere\\
Besøgt d. 08.10.2018 (Internet)\\

Kernekraft fortjener en ny chance. Udgivet af Politiken. Internetadresse:\\
politiken.dk/debat/arkiv\_ debattoerer/rasmusbrygger/art5502751/Kernekraft-fortjener-en-ny-chance\\
Besøgt d. 08.10.2018 (Internet)\\

Nuclear Power Reactors. Udgivet af World Nuclear Association. Internetadresse:\\
world-nuclear.org/information-library/nuclear-fuel-cycle/nuclear-power-reactors/nuclear-power-reactors.aspx\\
Besøgt d. 08.10.2018 (Internet)\\

Nye generationer af a-værker på vej. Udgivet af Ingeniøren. Internetadresse:\\
ing.dk/artikel/nye-generationer-af-vaerker-paa-vej-171605\\
Besøgt d. 08.10.2018 (Internet)\\

Technologie. Udgivet af International Thorium Energy Committee. Internetadresse:\\
ithec.org/en/technologie/\\
Besøgt d. 08.10.2018 (Internet)\\

Tjernobylulykken. Udgivet af Wikipedia. Internetadresse:\\
da.wikipedia.org/wiki/Tjernobylulykken\# Ulykken\\
Besøgt d. 08.10.2018 (Internet)\\

\section{Studierapport}

\subsection{AT 1 Klima}

Problemformulering:
“Hvordan kan det være, at forskeren Henrik Svensmark har så svært ved at få offentliggjort sit forskningsprojekt?”

Dansk:
Arbejdet med flg artikler:

Set filmen "Klimamysteriet"
Eigil Kaas og Jens Hesselbjerg: “10 nye myter i klimadebatten”
Eigil Kaas og Jens Hesselbjerg: “Tendensen er klar”
“Svensmark underbygger sin kontroversielle klimateori”
“Professor: Høje CO2-mængder får store konsekvenser.”
“Jordens egen skyld”

Metoder:
Appelformer og Toulmins argumentationsmodel, i : Langdahl, Olsen og Quist:  ”Krydsfelt”, Gyldendal, 2010 side 253 - 259
Argumenttyper og –kneb, i:Ole Schultz Larsen: ”Håndbog til dansk” Dansklærerforeningens forlag og Systime, 2015, side 161 – 167
Ciceros pentagram i: Larsen, 2015: ”Håndbog til dansk”, side 151 – 153

Fysik:
Litteratur: 
•	Ind i naturvidenskaben, kapitel 2 ”Videnskabsteori”
•	”Kosmiske partikler og skykim”, Uggerhøj, Pedersen og Enghoff, Aktuel Naturvidenskab(6), 2011
•	”Drivhusgasser – og deres betydning for klimaet”, Kaas og Langen, Aktuel Naturvidenskab(4), 2007

Eksperimentelt arbejde: 
•	Måling af forskellige materialers albedo

Andet:
•	Introduktion til SOLO-taksonomien

I AT forløbet har vi fundet ud af at Henrik Svensmark havde svært ved at få offentliggjort sit papir på hans teorier om global opvarmning. Vi har også nogenlunde fundet ud af hvorfor det var svært for ham. 

Ciceros pentagram og argumentationsanalyse gav os et godt overblik, samt analyse af hans teori og hans "modstanders" teori fulgte os til svaret for hvorfor han havde svært ved at få sin teori offentligtgjort.

\subsection{AT 2 Idræt og samfund}

Gruppeprojekt.

Matematik, idræt og historie.

Faglige mål 

– tilegne sig viden om en sag med anvendelse af relevante fag og faglige metoder

– foretage valg, afgrænsning og præcisering i arbejdet med sagen og på dette grundlag opstille og behandle en problemformulering samt selvstændigt fremlægge resultatet heraf

– perspektivere sagen

– vurdere de forskellige fags og faglige metoders muligheder og begrænsninger i forhold til den konkrete sag

– demonstrere indsigt i videnskabelig tankegang og gøre sig elementære videnskabsteoretiske overvejelser i forhold til den konkrete sag.

I AT2 forløbet har jeg fået meget ny information om både det entikke go det moderne OL, Kvinders kamp for rettigheder i det 19. og 20. århundrede. Og hvordan diktatorlande gennem tiderne har brugt OL til at sætte deres ideologier og styreformer ind i lyset.

Jeg har lært at bruge de forskellige metoder fra det forskellige fag som f.eks. kildekritik. Jeg har også lært at arbejde bedre og mere effektivt i grupper, som er noget jeg helt klart kan tage med mig til de næste AT forløb.

\subsection{AT 3 Fremtidens logist. udfordr. på transp.området}

FREMTIDENS LOGISTISKE UDFORDRINGER PÅ TRANSPORTOMRÅDET

Innovativt løsningsforslag til et problem vedrørende eksisterende og fremtidige transportformer.
HistorieA, MatematikA, FysikA, NaturgeografiC.

Faglige mål

– tilegne sig viden om en sag med anvendelse af relevante fag og faglige metoder

– foretage valg, afgrænsning og præcisering i arbejdet med sagen og på dette grundlag opstille og behandle en problemformulering samt selvstændigt fremlægge resultatet heraf

– perspektivere sagen

– vurdere de forskellige fags og faglige metoders muligheder og begrænsninger i forhold til den konkrete sag

– demonstrere indsigt i videnskabelig tankegang og gøre sig elementære videnskabsteoretiske overvejelser i forhold til den konkrete sag.

I AT3-forløbet har vi bruge fagende Historie A og Matematik A til at finde en løsning på problemet "Hvordan kan man løse problemet med kødannelser ved Limfjordsforbindelserne i myldretiden". Formålet med Forlåbet har været at finde en innovativ løsning på problemet. De metoder vi har brugt til at finde vores innovative løsning er statistisk beregning og historisk kildeanalyse og indholdsanalyse.

Vores innovative løsning er at vi bliver nødt til at bygge en tredje limfjordsforbindelse for at gøre noget ved trafikproblemerne i limfjordsforbindelserne, det er ikke nogen helt vild innovativ løsning og heller ikke nogen billig løsning, men det er den som vi har fundet som er mest langsigtet.

\subsection{AT 4 Videnskabelige gennembrud og teknologiske lan}

Videnskabelige gennembrud og teknologiske landvindinger 1851-1914

Du skal ud fra de overordnede problemstillinger for emnet Videnskabelige gennembrud og teknologiske landvindinger 1851-1914 udforme en problemformulering og skrive en synopsis, der kan danne udgangspunkt for den mundtlige prøve.

Du skal vælge et videnskabeligt eller teknologisk gennembrud i tidsrummet 1851-1914. Du skal undersøge dets fremkomst og belyse, i hvilken forstand der var tale om et gennembrud. Du skal diskutere, hvilke forudsætninger eller konsekvenser dette gennembrud havde eller fortsat har inden for kultur, samfund eller erkendelse.

Du skal anvende viden og metoder fra to af fagene fysik, matematik, tysk og dansk.

Faglige mål

– tilegne sig viden om en sag med anvendelse af relevante fag og faglige metoder

– foretage valg, afgrænsning og præcisering i arbejdet med sagen og på dette grundlag opstille og behandle en problemformulering samt selvstændigt fremlægge resultatet heraf

– perspektivere sagen

– vurdere de forskellige fags og faglige metoders muligheder og begrænsninger i forhold til den konkrete sag

– demonstrere indsigt i videnskabelig tankegang og gøre sig elementære videnskabsteoretiske overvejelser i forhold til den konkrete sag.

I AT4-forløbet har vi bruge fagende Dansk A og Fysik A til at finde en løsning på problemet "Hvilken betydning fik flyvning for menneskets syn på verden?". De metoder vi har brugt for at finde frem til vores konklusion er induktiv, deduktiv, litterær analyse og diskussion af forskellige fortolkninger af teksterne:"Om årtusinder" af H.C. Andersen og et uddrag af Karen Blixens bog:"Den afrikanske farm".

Vi har forsøgt at undersøge og belyse hvilke forudsætninger de først flyvepionere havde og hvilke teknologiske landvindinger muliggjorde deres flyvning. Ydermere har undersøgt, hvordan flyvning kom til udtryk i litteraturen, både inden og efter flyvning blev muliggjort. Vi kan konkludere at grunden til det lykkedes for Wright brødrene og ikke de andre var fordi at teknologien bare ikke helt var der for de andre. De prøvede at bruge dampmaskiner som simpelthen var for tunge. Mens Wright brødrene havde adgang til benzinmotorer. Derudover kan vi konkludere at flyvning blev udtrykt som noget fantastisk i litteraturen og man havde en forestilling om, inden flyvning blev muliggjort, at drivkraften ville være dampkraft.

\subsection{AT 5 årsprøve Menneskets forhold til naturen}

I dette AT5-forløb har vi brugt matematik på a og fysik på a for at besvare spørgsmålende

Hvordan har newtons love påvirket forståelsen af naturen og hvilken betydning har matematikken haft i forhold til Newtons love? 
Hvordan kom Newton frem til sine love? 
Hvorfor er Newtons love relevante for en bedre forståelse af naturen?
I hvilke større begivenheder er newtons love brugt i praksis?

Vi kan konkludere, at menneskets forståelse af naturen har ændret sig drastisk med udviklingen af Newtons love om bevægelse, i den forstand at hans tre love har givet en bedre forståelse og vurdering af de kræfter der påvirker os og alt omkring os. Ved hjælp af Newtons love har vi også fået redskaber, som gør at vi ved brug af matematik kan udregne præcise kræfter for, hvad der foregår. F.eks. hvor et skud fra en kanon vil lande, hvor lang tid det tager et objekt at ramme jorden osv. Før Newtons love ville det meste af sådanne beregninger være enten umulige eller meget upræcise. På den måde har Newtons love givet en forståelse for de love som naturen opfører sig efter, og hvordan vi ved hjælp af disse ubrydelige love kan forudsige scenarier. 

I forhold til den matematiske del, kan vi se at, som matematik har udviklet sig har forståelsen af fysik også. I Newtons love indgår differentialligninger, som blev udviklet af Newton og derfor ikke fandtes førhen. Aristoteles primitive forståelse kan der igennem også blive tilkoblet den mere primitive matematik han havde til rådighed. Langt hen ad vejen har Aristoteles har ofte draget sine konklusioner ud fra gæt uden understøttende beviser, mens Newton har haft mulighed for at kunne underbygge sine hypoteser med forsøg, og matematiske beviser, som resultere i en mere fuldstændig forståelse uden ubegrundede gæt.

\subsection{AT6}

AT6 - Atomreaktorer

Problemformulering
Hvordan blev atomkraft udviklet som fredelig energiteknologi - og hvad er denne teknologis fremtid?

Konklusion
For at konkludere har vi fundet at atomkraftens har undergået en stor udvikling, fra det stadie som mange tænker på når de hører ordet atomkraft. De fleste tænker på Tjernobyl når de tænker på atomkraft og dermed også stor fare og usikkerhed, men som fredelig energiteknologi har atomkraftværkerne udviklet sig meget og der er kommet flere nye generationer der har gjort chancen for endnu en ulykke til kun en teoretisk en. På den måde har atomkraften fået et unfair ry, hvor spørgsmålet om atomkraftværk som en mulig energikilde ikke længere bliver bragt op i dansk politik. Selvom det i dag er noget helt andet end hvad det var da man i 1985 indførte loven imod inkluderingen af atomenergi. Det er blevet meget sikrere, mere effektivt og mere miljøvenligt, og fremtiden ser kun endnu bedre ud for atomkraft. Med forskning indenfor thorium-reaktorer, med bedre sikkerhed, mindre radioaktivt affald og ingen sekundær udnyttelse inden for krigsføring. Derfor er det muligt at forestille sig at debatten om hvorvidt Danmark skal have atomkraftværker vil komme op igen engang i fremtiden

I Historie-delen har vi brugt kildekritik for at kunne se på om de kilder vi har brugt er troværdige eller ej. Derudover har vi kigget på nogle historiske a-værker og hvordan de virker, og kigget på hvorfor det gik så galt i Tjernobyl i 1986 og så brugt metoder indenfor fysik for at kunne finde ud af hvad der bliver forsket i lige nu for at sådan en situation aldrig kommer til at forekomme igen.

\end{document} 
