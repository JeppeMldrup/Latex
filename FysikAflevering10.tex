\documentclass[12pt]{article}

\usepackage[utf8]{inputenc}
\usepackage{amsmath}
\usepackage{fancyhdr}
\usepackage{graphicx}
\usepackage{vmargin}
\usepackage{chemfig}
\usepackage{tikz}
\usepackage{pgfplots}

%%%%%%%%%%%%%%%%%%%%%%%%%%%%%%%%%%%%%%%%%%%%%%%%%%%%%%%%%%%%%%%%%%%%%%%%%%

\setmarginsrb{3 cm}{2.5 cm}{3 cm}{2.5 cm}{1 cm}{1.5 cm}{1 cm}{1.5 cm}
\pagestyle{fancy}
\fancyhf{}
\rhead{18-04-2018}
\chead{Fysik Aflevering 10}
\lhead{Jeppe Møldrup}
\rfoot{side \thepage}

%%%%%%%%%%%%%%%%%%%%%%%%%%%%%%%%%%%%%%%%%%%%%%%%%%%%%%%%%%%%%%%%%%%%%%%%%%

\begin{document}

\Large{\textbf{Fysik Aflevering 10}}
\normalsize

\section*{Opgave 1.}
\begin{enumerate}
  \item[a.] Jeg starter med at finde arealet af lugen. Da jeg ved hvad radius er kan jeg bruge formlen
  $$T = \pi \cdot r^2$$
  Med mine værdier
  $$T = \pi \cdot (0.21 m)^2 = 0.14 \ m^2$$
  Nu finder jeg rumfanget af den søjle af vand som er over lugen med formlen
  $$V = G \cdot h$$
  Med mine værdier
  $$V = 0.14 \ m^2 \cdot (10911-2) \ m = 1527.26 \ m^3$$
  Jeg beregner min højde af søjlen som $(10911-2) \ m$ da bunden af kabinen ligger ved $10911 \ m$ og jeg antager at kabinen er omkring $2 \ m$ høj\\
  Nu kan jeg finde tyngdekraften af vandsøjlen
  $$1527.26 \ m^3 \cdot 1060 \ \frac{kg}{m^3} \cdot 9.82 \ \frac{m}{s^2} = 15732247.02 \ N \approx 15.73 \ MN$$

  \item[b.] Da jeg ved at gennemsnitlige densitet for kabinen skal være den samme som havvandet, altså $1060 \ \frac{kg}{m^3}$.
  Så starter jeg med at finde hvor stor en kraft skummet skal udligne.
  Det gør jeg ved at udregne hvor stor en tyngdekraft kabinen har uden skum og trække kabinens opdrift fra.
  $$6500 \ kg \cdot 9.82 \ \frac{m}{s^2} = 63830 \ N$$
  $$3.29 \ m^3 \cdot 1060 \ \frac{kg}{m^3} \cdot 9.82 \ \frac{m}{s^2} = 34246.27 \ N$$
  $$63830 \ N - 34246.27 \ N = 29583.73 \ N$$
  Så skummet skal udligne en kraft på omkring $29583.73 \ N$ så jeg bruger arkimedes lov og isolere rumfanget
  $$F_{opdrift} = \rho \cdot V \cdot g$$
  Med mine værdier
  $$29583.73 \ N = 1060 \ \frac{kg}{m^3} \cdot V \cdot 9.82 \ \frac{m}{s^2} $$
  $$\leftrightarrow $$
  $$ V = \frac{29583.73 \ N}{1060 \ \frac{kg}{m^3} \cdot 9.82 \ \frac{m}{s^2}} = 2.84 \ m^3$$
  Så rumfanget af skummet skal være omkring $2.84 \ m^3$ hvis den gennemsnitlige densitet af kabinen skal være lig med havvandet
\end{enumerate}

\section*{Opgave 2.}
\begin{enumerate}
  \item[a.] $^{177}Lu$ henfalder med $\beta ^{-}$-henfald dvs. at en neutron omdannes til en proton og udsender en elektron on en antineutrino
  så reaktionsskemaet vil derfor være
  $$_{71}^{177}Lu \rightarrow _{72}^{177}Hf + ^{-1}e + \overline{v}_{e}$$

  \item[b.] Da jeg kender aktiviteten og henfaldskonstanten for indsprøjtningen kan jeg bruge formlen
  $$A = k \cdot N \leftrightarrow N = \frac{A}{k}$$
  Så jeg indsætter mine værdier
  $$N = \frac{5.5 \ GBq}{0.000029 \ s^{-1}} = 189874 \cdot 10^{9} \ antal \ kerner$$
  Massen af en kerne er $176.9437581 \ U$ så massen af indsprøjtningen vil være
  $$189874 \cdot 10^{9} \ kerner \cdot 176.9437581 \ U = 33597019 \cdot 10^{9} \ U$$

  \item[c.] Jeg starter med at udregne hvor mange henfald der sker indenfor et døgn med formlen
  $$N = N_0 \cdot \left( \frac{1}{2} \right)^{\frac{t}{T_{1/2}}}$$
  med mine værdier
  $$A = 189874 \cdot 10^{9} \cdot \left( \frac{1}{2} \right)^{\frac{1}{6.647}} = 189854 \cdot 10^{9} \ kerner$$
  Så finder jeg $\Delta N$ da det er antal henfald
  $$189874 \cdot 10^{9} - 189854 \cdot 10^{9} = 20 \cdot 10^{9} \ kerner$$
  Så ganger jeg det med energi per henfald for at få den samlede energi
  $$20 \cdot 10^{9} \ kerner \cdot 2.2 \cdot 10^{-14} \ \frac{J}{kerner} = 0.00044 \ J$$
  Så den energi der bliver afsat i personen efter det første døgn er omkring $0.00044 \ J$
\end{enumerate}

\section*{Opgave 3.} Jeg starter med at finde ud af hvor meget energi det tager at nedkøle zinkpladen fra 24 grader til 5.8 grader med ligningen
$$Q=m \cdot C \cdot \Delta T$$
Jeg indsætter mine værdier
$$Q = 0.0864 \ kg \cdot 25.390 \ \frac{J}{kg \cdot K} \cdot (24-5.8) \ K = 39.925 \ J$$
Jeg antager at kuldesprayen er ved sit kogepunkt når det forlader flasken. Så derfor skal der tilføres $39.925 \ J$ til $2.1 \ g$ kuldespray for at fordampe det\\
Men jeg skal finde fordampningsvarmen. Så jeg opstiller en ligning med formlen
$$Q = m \cdot L$$
Hvor m er massen, L er fordampningsvarmen og Q er energi.
Så jeg indsætter mine værdier og finder L for kuldesprayen
$$L=\frac{39.925 \ J}{2.1 \ g} = 19.012 \ \frac{kJ}{kg}$$

\end{document}
