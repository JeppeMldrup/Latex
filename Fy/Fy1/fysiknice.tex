\documentclass[12 pt]{article}
\usepackage[utf8]{inputenc}
\usepackage{amsmath}
\usepackage{fancyhdr}
\usepackage{vmargin}
\usepackage{graphicx}
\usepackage{caption}
\usepackage{chemfig}

\title{Atwoods faldmaskine}
\author{Jeppe, Christian, Mads og Marius 3.X}
\date

\begin{document}
\maketitle
\section{Formål}
Vi vil gå ind og undersøge Newtons 2.lov og mekanisk energibevarelse ved at lave Atwoods faldmaskine.
\section{Teori}
Newtons 2.lov. Den accelererede masse er den samlede masse. Den resulterende kraft er tyngdekraften på masseforskellen af de to lodder (vis dette i rapporten).
Vis at Newtons 2.lov fører frem til at accelerationen er givet ved:

\section{Apparatur}
Capstone med interface. Smart Pulley, tråd og to lodholdere med tilhørende lodder. For at kunne se bort fra gnidning i trissen (Smart Pulley)
skal massen af de to lodder være ca. $70 g$ hver.

\section{Udførelse}
Smart Pulley hænges op i et stativ, så lodderne kan falde frit. Smart Pulley tilsluttes Pasco interfacet. I Capstone tilsluttes Smart Pulley til den relevante indgang.
Programmets valg af måleopsætning fungerer udmærket. Forsøget falder i tre dele:
\begin{enumerate}
  \item Der er en lille masseforskel på de to lodder (ca. $10 g$). Hvor vi finder et udtryk for $s(t)$ og $v(t)$ (dette gøres ved hjælp af programmet).
  \item Konstant masse: Lodder flyttes fra den ene lodholder til den anden.
  \item Fast masseforskel svarende til fast resulterende kraft. Der tages lodder at begge steder.
\end{enumerate}
For hver kørsel tegnes positions og hastighedsgrafer og accelerationen findes ved hjælp af hastighedsgrafen.
 Accelerationsgrafen er ikke særlig køn fordi Capstone beregner den udfra stedfunktionen (Hvilket er det eneste som Pasco udstyret har målt).
 Så dermed er accelerationsgrafen ikke særlig køn fordi det ikke er en målt acceleration men en beregnet acceleration udfra et sted som er med at gøre det upræcist.
\section{Databehandling}

\end{document}
