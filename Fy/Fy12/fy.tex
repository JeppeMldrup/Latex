\documentclass[12pt]{article}

\usepackage[utf8]{inputenc}
\usepackage{fancyhdr}

\pagestyle{fancy}
\fancyhead[L]{Jeppe Møldrup}
\fancyhead[C]{Fysik 13}
\fancyhead[R]{12/4-2019}

\title{Fysik 13}
\author{Jeppe Møldrup}
\date{}

\begin{document}

\maketitle{}

\section*{Opgave 1}

a

Jeg finder frekvensen ud fra bølgelængden 1061.6 nm med formlen
$$c = \lambda \cdot f \Leftrightarrow f = \frac{c}{\lambda} \Leftrightarrow f = 2.824 \cdot 10^{14}\ s^{-1}$$
Så frekvensen af den observerede spektrallinje er 2.824$\cdot 10^{14}\ s^{-1}$

b

Jeg bruger formlen
$$z = \frac{modtaget-udsendt}{udsendt} = \frac{1061.6\ nm-121.6\ nm}{121.6\ nm} = 7.73$$
Så rødforskydningen er 7.73

\section*{Opgave 2}

a

Jeg finder den maksimale spænding og udregner strømstyrken med formlen
$$U=IR \Leftrightarrow I = \frac{U}{R}$$
$$U_{max} = 8.5\ V ~~~~ R = 2.7\ \Omega$$
$$I = \frac{8.5\ V}{2.7\ \Omega} = 3.148 A$$
Så den maksimale strømstyrke i kredsløbet er 3.1 A

b

Jeg aflæser tiden det tager for de 4 ryst til at være 0.16 sekunder, og den tilbagelagte afstand er $4\cdot 60\ mm=240\ mm$,
dvs. gennemsnitshastigheden vil være
$$\frac{240\ mm}{0.16\ s} = 0.38\ m/s$$
Så gennemsnitshastigheden af magneten er 0.38 m/s under rystelserne.

c

Faradays lov siger at
$$U_{ind} = -\frac{d\Phi}{dt}$$
Og jeg skal finde flux til tiden 0.06 s og trække flux til tiden 0.04 s fra, dvs. jheg skal finde arealet under grafen fra 0.04 s til 0.06 s.
Da jeg ikke kender funktionsforskriften tæller jeg tern. 1 tern = $-1\frac{Wb}{s}\cdot 0.1\ s = -0.1\ Wb$. Jeg vurderer arealet til ate være 13
tern dvs. -1.3 Wb.\\
Så ændringen i magnetisk flux fra 0.04 sekunter til 0.06 sekunter er -1.3 Wb.

\section*{Opgave 3}

a

Jeg bruger formlen
$$P_{elektrisk} = U\cdot I \Leftrightarrow P_{elektrisk} = 230\ V \cdot 4.0\ mA = 920\ mW$$
Så opladeren omsætter energi med effekten 920 mW

b

Faradays lov siger
$$U_{ind} = -\frac{d\Phi}{dt}$$
Så jeg finder hældningen af grafen til punktet t = 0.010 ms. Jeg finder de to punkter (0.09 ms, 0.125 mT) og
(0.11 ms, -0.0125 mT) hvor hældningen så ville være
$$\frac{-0.250\ mT}{0.02\ ms} = -12.5\ T/s = 12.5\ V.$$
Så spændingsfaldet der er induceret i tandbørsten til tidspunktet t = 0.010 ms er 12.5 V.

\section*{Opgave 4}

a

Jeg benytter formlen
$$n\cdot \lambda = d\cdot sin(\phi) \Leftrightarrow \phi = sin^{-1} \left( \frac{n\cdot \lambda}{d} \right)$$
Gitterkonstanten findes d = 1/800 mm.\\
Så jeg indsætter
$$\phi = sin^{-1}(520.16\ nm \cdot (800\ mm)) = 24.188^{\circ}$$
Så afbøjningsvinklen til 1. orden er 24.2 grader.

b

Jeg benytter formlen
$$z = \frac{modtaget-udsendt}{udsendt} \Leftrightarrow z = \frac{520.16\ nm-486.16\ nm}{486.16\ nm} = 0.07$$
Idet solen ikke flyver væk fra os har den ingen rødforkydning og derfor kan jeg bruge dens bølgelængde som den udsendte af BAS11.\\
Så bruger jeg formlen
$$z = \frac{v}{c} \Leftrightarrow v = z\cdot c = 0.07\cdot c = 21000\ km/s$$
Og bruger formlen
$$v = H\cdot D \Leftrightarrow D = \frac{v}{H} = \frac{21000\ km/s}{H} = 9\cdot 10^{24}\ m$$
Så afstanden til BAS11 er $9\cdot 10^{24}$ m.

\end{document}
